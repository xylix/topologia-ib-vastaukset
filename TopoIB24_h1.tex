\documentclass[12pt,a4paper,leqno]{amsart}
%\documentclass[12pt,a4]{article}   %%% vaihtoehtoinen formatti
\usepackage[finnish]{babel}
\usepackage[T1]{fontenc}
\usepackage{lmodern}
\usepackage{amsmath}
\usepackage{amssymb}
\usepackage{amsfonts}
\addtolength{\hoffset}{-1.75cm}
\addtolength{\textwidth}{3.5cm}
\addtolength{\voffset}{-1cm}
\addtolength{\textheight}{2cm}
\setlength{\parindent}{0pt}
\newcommand{\css}{\operatorname{\subset\!\!\!\!_{^{^c}}}}
\newcommand{\oss}{\operatorname{\subset\!\!\!\!_{^{^\circ}}}}
% tää ei toimi \newcommand{\internal}{\operatorname{\int}}

%% sivuasetuksia
%\pagestyle{empty}
%\thispagestyle{empty}

\begin{document}

\noindent MAT21006 Topologia IB

\noindent Harjoitus 1 

\noindent  Määräaika:  \textbf{torstaina 21.3.2024 klo 23.00 mennessä}.

\bigskip


Aihepiiri:  9. \textit{Homeomorfismit}.  Viite [V,9.16] tarkoittaa  kohtaa 9.16 Väisälän kirjasta.

\bigskip

\noindent  1:1. Määritellään kuvaus 
$f: \mathbb R^2 \to \mathbb R^2$ kaavalla
\[
f(x,y) = (x,y+\sin(x)), \quad (x,y) \in \mathbb R^2.
\]
Näytä, että $f$ on homeomorfismi  $\mathbb R^2 \to \mathbb R^2$. Tasossa $\mathbb R^2$ on euklidinen metriikka.
\\
\textbf{Ratkaisu:}
$f$:n ensimmäinen koordinaatti on selkeästi jatkuva koska identiteettifunktio on jatkuva. Tarkastellaan $f$:n toista koordinaattia: Se koostuu identiteettifunktiosta ja $\sin(x)$:stä, jonka derivaatta y:n suhteen on 0. Se on siis jatkuva. Sillä on myös käänteisfunktio $(y - cos(x)$. Käänteisfunktio on samalla perustelulla jatkuva. Täten $f$ on homeomorfismi $\mathbb R^2 \to \mathbb R^2$


\noindent  1:2.  Olkoon 
\[
f(x,y) = (x,e^{-x}y), \quad (x,y) \in \mathbb R^2,
\]
ja tasossa $ \mathbb R^2$ euklidinen metriikka. 

\begin{enumerate}
\item Näytä, että $f$ on homeomorfismi $\mathbb R^2 \to  \mathbb R^2$, ja määrää
käänteiskuvauksen $f^{-1}$ lauseke. 
\smallskip
\textbf{Ratkaisu: } 
$f(x, y) = (x, y)$ olisi identiteettifunktiona homeomorfismi. y:stä riippumattoman jatkuvan kertoimen lisääminen säilyttää homeomorfismiominaisuudet..

Käänteisarvoja pohdintaa helpottamaan: 
(0, y) = (0, y)
(1, y) = (1, y/e)
$(2, y) = (2, y/e^2)$
(3, y), = ...
ratkaistaan yhtälö: $(x, y) = f(f^{-1}(x, y)) = $
$e^{-x} y = \frac{1}{e^x} y \iff  $
Käänteiskuvaus: Havaitaan että $e^{-x}y$ käänteiskuvauksen täytyy sisältää e kantainen eksponenttifunktio tai logaritmi. Päätellään siis: $f^{-1}(x, y) = (x, -\ln(y))$

Tarkistetaan: $f(f^{-1}(x, y)) = f(x, -\ln (y) ) = (x, e^{-(-\ln(y))}y) = (x, y)$

\item Tarkastellaan suoria $A =\{(x,0): x \in \mathbb R\}$ 
ja $B =\{(x,1): x \in \mathbb R\}$.
Näytä, että etäisyys $d(f(A),f(B)) = 0$.   
\end{enumerate}

Joukkojen välinen etäisyys on määritelty kohdassa [V,2.9]. Tehtävän perusteella kyseinen  etäisyys ei
ole topologinen ominaisuus (eli se ei aina säily homeomorfismeissa).
\\
\textbf{Ratkaisu:}
Veikataan että lähimmät pisteet fA:ssa ja fB:ssä ovat piste $f(0, 0)$, ja piste (x, 1) B:stä kun $\lim_{x \to \infty}$ . Lasketaan pisteiden arvot: 
\[B: \lim_{x \to \infty} (x, e^{-x}1 = \frac{1}{e^{\infty}}) = 0\]
Lasketaan etäisyys: 
\begin{align*}
& d(f(A), f(B)) = \inf \{ d(x, y)) : x \in fA, y \in fB\} \\
& d(f(0, 0), \lim_{y \to \infty} f(0, y)) \\
& = d((0, 0), (0, 0)) = 0 \\
& \Rightarrow \inf \{ d(x, y)) : x \in fA, y \in fB\} = 0
\end{align*}
Etäisyys on siis 0.
\bigskip

\noindent  1:3. Olkoon $\vert \cdot \vert$ itseisarvometriikka reaalisuoralla $\mathbb R$, sekä $e$ reaalisuoran diskreetti $\{0,1\}$-metriikka.
Näytä, että metriset avaruudet $(\mathbb R, \vert \cdot \vert)$ ja $ (\mathbb R,e)$ eivät ole homeomorfisia.
\textit{Vihje}:  tee vastaoletus, eli
on olemassa homeomorfismi $g:   (\mathbb R, \vert \cdot \vert) \to (\mathbb R,e)$. (Muitakin ratkaisutapoja on olemassa.)
\\
\textbf{Ratkaisu:} 
Tarkastellaan vastaoletuksen kautta: Jotta $(\mathbb R, \vert \cdot \vert)$ ja $ (\mathbb R,e)$ voisivat olla homeomorfisia, tulee olla olemassa jokin $g:(\mathbb R, \vert \cdot \vert) \to (\mathbb R,e)$ joka on homeomorfismi.

Tällaisen g:n täytyisi täyttää kaikki kolme homeomorfismin ehtoa: Bijektio, jatkuvuus, käänteisfunktion jatkuvuus.

Homeomorfismi säilyttää joukon avoimuuden ja suljettuuden (V. 9.5.3 3' ja 3''): 
$3':  U \oss X \Rightarrow fU \oss fX$
$3'':  F \css X \Rightarrow fF \css fX$

Diskreetin $\{0, 1\}$ metriikan määrämä avaruudessa $(\mathbb R,e)$ kaikki joukot ovat sekä avoimia että suljettuja. 

Esimerkiksi avoin, ei suljettu joukko (0, 1) kuvautuisi avaruuteen $(\mathbb{R}, e)$ avoimena ja suljettuna. Tällöin homeomorfisuus ei toteudu.

% TODO: tsekkaa 9.15 ja 9.16 väisälä. Tässä on varmaan joku gotcha. Käytännössä: pitää löytää jokin topologinen ominaisuus jota mikään funktio g ei voi säilyttää (\mathbb R, \vert \cdot \vert) \to (\mathbb R,e)$ yli!

\bigskip

\textit{Tehtävän 1:4  tarkoituksena on  osoittaa seuraavat kohdan [V,9.16]  tiedot. 
Yleisperiaate [V,9.15] ei ole todistettu monisteessa tai luennoilla, eli siihen ei voi vedota.}

\bigskip

\noindent  1:4. Olkoon $(X,d)$ ja $(Y,d')$ metrisiä avaruuksia, sekä 
$f: X \to Y$ homeomorfismi. 
Näytä, että jokaisella osajoukolla  $A \subset X$ on voimassa 

\begin{enumerate}
\item $f(\overline{A}) = \overline{f(A)}$
\textbf{Ratkaisu: }
Lauseen 6.13 mukaan kaikilla joukoilla $A \subset X$:
\begin{align*}
f\overline{A} & \subset \overline{fA} \\
\end{align*}
Täytyy siis osoittaa toinen suunta, eli:
\begin{align*}
& \overline{fA} \subset f\overline{A}  \\
& 
\end{align*}
Hyödynnetään lausetta 6.8 (3):
% TODO: fix:
Jos $A \subset B \css X$, niin $\overline{A} \subset B$
En nyt saanut tätä paremmin ratkaistua, pahoittelut tarkistajalle.
.
\smallskip

\item $f(\partial A) = \partial (f(A))$.

Edellä $\overline{A}$ on joukon $A$ sulkeuma ja $\partial A$ sen reuna.  \textit{Muistutus}: kohta [V,6.13] on hyödyllinen.

\end{enumerate}
\textbf{Ratkaisu:}
% TODO: kesken, korjaa tai muuta palautettavaan muotoon

Mikäli A, ja siten $f(A)$ suljettu:
\begin{align*}
    A \css X \iff \partial A \subset A  \\
    \Rightarrow \\
    \Rightarrow \partial A \\
     &f(\partial A) = \partial (f(A)) \\
    \iff & 
\end{align*}
Mikäli A (ja siten f(A) avoin (8.2.1: (2), (5) (7)):
\begin{align*}
    A \oss X \iff int A = A \\
    A \oss X \Rightarrow \partial A = \overline{A} \setminus A \\
    \partial A = \overline{A} \setminus intA
\end{align*}
(toimii vain avoimille)
Koska 1 kohdassa todistettiin että $f(\overline{A}) = \overline{f(A)}$, voidaan kirjoittaa kaavan $\partial A = \overline{A} \setminus intA$ avulla väite muotoon: 
\begin{align*}
    f(\partial A) = \partial (f(A)) \\
\iff f(\overline{A} \setminus int A) = \overline{fA} \setminus int(fA) \\
\end{align*}

\bigskip

\noindent  1:5.  Olkoon $g(x) = (x,\cos(x))$, kun $x \in \mathbb R$. 

\begin{enumerate}
\item  Onko $g$ $M$-bilipschitz kuvaus  $\mathbb R \to \mathbb R^2$ jollakin vakiolla $M \ge 1$?  \textit{Muistutus}: 
differentiaali\-laskennan väliarvolause. 

\textbf{Ratkaisu:}
Tarkastellaan:
Testataan onko g bilipschitz vakiolla $M \geq 1$:
\\
$d = \sqrt{(x - y)^2}$
$d' = (\sum^2_{i=1}((x_i - y_i)^2))^{\frac{1}{2}} = \sqrt{(x_1 - y_1)^2 + (x_2 - y_2)^2}$  

\begin{align*}
    d(i, j) / M \leq d'(f(i), f(j)) \leq M d(i, j) \\
    \iff d(i, j) / M \leq d(i,\cos(i)), (j,\sin(j))) \leq M d(i, j) \\ 
    \iff \sqrt{(i - j)^2} / M \leq \sqrt{((i - j)^2 + (\cos(i) - \cos(j))^2)}\leq M \sqrt{(i - j)^2} \\ 
    \text{Neliöjuuret ovat positiivisia joten voidaan korottaa yhtälö toiseen potenssiin} \\
    \iff (i - j)^2 \cdot \frac{1}{M^2} \leq (i - j)^2 + (\cos(i) - \cos(j))^2\leq (i - j)^2 M^2  \\ 
    \iff (i - j)^2 \cdot (\frac{1}{M^2} - 1) \leq (\cos(i) - \cos(j))^2\leq (i - j)^2 (M^2 - 1) \\
\end{align*}
Tarkastellellaan päteekö epäyhtälö $cos(i) - cos(j)$ 

Kokeillaan arvoja $i = 2n\pi, j=(n+1)\pi$
\begin{align*}
        \sqrt{(2n\pi - (2n+1)\pi)^2} / M \leq \\\sqrt{((2n\pi - (2n+1)\pi)^2 + (\cos(2n\pi) - \cos((2n+1)\pi))^2)} \\
    \leq M \sqrt{(2n\pi - (2n+1)\pi)^2} \\ 
    \iff \sqrt{(-\pi)^2} / M \leq \sqrt{((-\pi)^2 + (1 - 1))^2)} \leq M \sqrt{(-\pi)^2} \\ 
\end{align*}
Kokeillaan arvoja $i=2n\pi,j= \pi$
\begin{align*}
        \sqrt{((2n-1)\pi)^2} / M \leq \sqrt{((2n-1)\pi)^2 + (1 + 1)^2}  \leq M \sqrt{((2n-1)\pi)^2} 
\end{align*}
% TODO: kokeile x1=2nπ,x2=(2n+1)π,
% TODO sovella väliarvolausetta
Ei ratkennut tästä enempää tänä iltana. 
\smallskip

\item Onko $g$ upotus $\mathbb R \to \mathbb R^2$?
\end{enumerate}
\textbf{Ratkaisu:} 
$g$ ei ole bijektiivinen koska sen y-koordinaatit eivät ole surjektiivisia, joten se ei ole myöskään upotus.
\bigskip

\noindent  1:6.   Olkoon $d$ ja $e$ metriikkoja joukossa $X$. Näytä, että
$\tau_d \subset \tau_e$ jos ja vain jos jokaista $a \in X$ ja jokaista $r > 0$
kohti on olemassa sellainen $s > 0$, että
\[
B_e(a,s) \subset B_d(a,r).
\]
Tässä 
\[
\tau_d = \{U \subset X: U\  \textrm{on avoin avaruudessa}\ (X,d)\}
\]
 on  metrisen avaruuden $(X,d)$ $d$-topologia.
\textit{Vihje}: kohdan [V,3.7]  todistusideasta on hyötyä.

\textbf{Ratkaisu: }
Väite on siis että $T_d \subset T_e$ joss $B_e(a,s) \subset B_d(a,r)$. Toisin sanottuna avoimet topologia $T_d$ ovat alajoukko $T_e$:stä joss jokaiselle avoimelle kuulalle $B_e$ on olemassa jokin avoin kuula $B_d$ johon se sisältyy.

$T_d \subset T_e$ pätee ainoastaan jos ja vain jos jokainen joukko $T_d$:ssä kuuluu $T_e$:hen. Jotta $\forall d \in T_d: d \in T_e$, niin tällöin jokaista $d \in T_d$ kohti on olemassa jokin $B_e$ pallo joka sisältyy kaikkien $B_d$ pallojen joukkoon. Toisin sanottuna täytyy olla $B_e(a,s) \subset B_d(a,r)$, mikä piti todistaa.

\bigskip

Kaikissa tehtävissä vastaukset tulee \textbf{perustella}. Pelkkä vastaus ei riitä pisteisiin, ellei tehtävässä erityisesti niin mainita.
Tehtävien arvostelussa käytetään seuraavaa asteikkoa: 
\begin{itemize}
\item [0 p.] Ei ratkaisua tai ratkaisussa ei oikeita elementtejä.
\item [1 p.] Ratkaisussa oikeita elementtejä, mutta kokonaisuutena puutteellinen tai vain osa tehtävästä on ratkaistu.
\item [2 p.] Ratkaisu (lähes) oikein.
\end{itemize}


\end{document}









 
