\documentclass[12pt,a4paper,leqno]{amsart}
%\documentclass[12pt,a4]{article}   %%% vaihtoehtoinen formatti pienempi oletuskoko
\usepackage[finnish]{babel}
\usepackage[T1]{fontenc}
\usepackage{lmodern}
\usepackage{amsmath}
\usepackage{amssymb}
\usepackage{amsfonts}
\addtolength{\hoffset}{-1.75cm}
\addtolength{\textwidth}{3.5cm}
\addtolength{\voffset}{-1cm}
\addtolength{\textheight}{2cm}
\setlength{\parindent}{0pt}
%% sivuasetuksia
%\pagestyle{empty}
%\thispagestyle{empty}
\usepackage{comment}
\newcommand{\css}{\operatorname{\subset\!\!\!\!_{^{^c}}}}
\newcommand{\oss}{\operatorname{\subset\!\!\!\!_{^{^\circ}}}}
% tää ei toimi \newcommand{\internal}{\operatorname{\int}}


\begin{document}

\noindent MAT21006 Topologia IB

\noindent Harjoitus 2 
\begin{comment}
\noindent  Määräaika:  \textbf{torstaina 4.4.2024 klo 23.00 mennessä} (pääsiäisviikko)

\bigskip

\noindent Ratkaisut palautetaan sähköisesti määräaikaan mennessä kurssin Moodle sivun viikottaiseen palautusalueeseen yhtenä pdf-lähetyksenä. 
%% Muista itse-rekisteröityä Moodle sivulle (muuten palautusalue ei ehkä aukene).

\smallskip

Pääsiäisen aikataulun takia ohjausvuorot  laskuharjoitustehtävien ratkaisemiseksi  pidetään

\smallskip

\textbf{torstaina 4.4 klo 12-14, ja klo 14-16} Ratkomossa  (kurssin Topologia IB ohjaaja).

\smallskip

Sen sijaan ohjausvuoroa \textbf{ei pidetä tiistaina 26.3}.
Muina aikoina voi  kysyä  neuvoja tehtäviin Ratkomossa tai kurssin Moodle sivun Keskustelualueissa. Kurssin Topologia IA
Telegram-ryhmä jatkaa. 

\bigskip

Aihepiiri:  Luku 10. \textit{Metriikkojen ekvivalenssi. Tuloavaruus}.  Luku 11. \textit{Pistejonot ja raja-arvot}, kohdat [V,11.1-11.5]
Väisälän kirjasta. 
\end{comment}
\bigskip

\noindent  2:1. Merkitään $d \buildrel{bL}\over\sim e$ kun metriikat $d$ ja $e$ ovat bilipschitz-ekvivalentit 
annetussa joukossa $X$ (kohta  [V,10.3]). Näytä, että metriikkojen bilipschitz-ekvivalenssi $d \buildrel{bL}\over\sim  e$
on ekvivalenssirelaatio joukon $X$ kaikkien metriikkojen kokoelmassa.
\\
\textbf{Ratkaisu: }
Bilipschitz-ekvivalenssin määritelmän mukaan (V, 10.3): 
$a \buildrel{bL}\over\sim b \iff \mathop{a}(x, y)/M \leq \mathop{b}(x, y) \leq M \mathop{a}(x, y)$, $M \geq 1$
Tarkastellaan ekvivalenssirelaation määritelmällä:
\begin{enumerate}
    \item $a\mathop{R}a$ pätee \\

$a \buildrel{bL}\over\sim a \iff \mathop{a}(x, y)/M \leq \mathop{a}(x, y) \leq M \mathop{a}(x, y)$ yhtälö pätee M=1, tällöin vertailut ovat yhtäsuuria.

    \item jos $a\mathop{R}b$ niin myös $b\mathop{R}a$ \\
    $a \buildrel{bL}\over\sim b \implies b \buildrel{bL}\over\sim a$ Tiedetään vasen puoli, johdetaan oikea: 
\begin{align*}\\
    a \buildrel{bL}\over\sim b \iff \mathop{a}(x, y)/M & \leq \mathop{b}(x, y) \leq M \mathop{a}(x, y) 
\\ \implies \mathop{a}(x, y)/M & \leq \mathop{b}(x, y) \leq M \mathop{a}(x, y) \leq M_2 b(x, y)
\\ \implies \mathop{b}(x, y) & \leq M \mathop{a}(x, y) \leq M_2 b(x, y)
\text{jaetaan M, joka yli 1}
\\ \implies \mathop{b}(x, y) /M & \leq \mathop{a}(x, y) \leq M_2 \frac{b(x, y)}{M} \text{ nimetään tätä *}
\end{align*}
Valitaan $M_2 \geq M_1$ jolloin pätee:
\begin{align*}
b(x, y) /M \leq b(x, y) / M_2 &
\\ M_2 \frac{b(x, y)}{M} \leq M_2 b(x, y) &
\\ * \implies \mathop{b}(x, y) /M_2 & \leq \mathop{a}(x, y) \leq M_2 b(x, y)
        \end{align*}
Joka piti osoittaa.

    \item Jos $a\mathop{R}b$ ja $b\mathop{R}c$ niin myös $a\mathop{R}c$
$a \buildrel{bL}\over\sim b \land b \buildrel{bL}\over\sim c \implies a \buildrel{bL}\over\sim c$
Kun pätee: 
\begin{align*}
a \buildrel{bL}\over\sim b & \iff \mathop{a}(x, y)/M_1 \leq \mathop{b}(x, y) \leq M_1 \mathop{a}(x, y) \\
b \buildrel{bL}\over\sim c & \iff \mathop{b}(x, y)/M_2 \leq \mathop{c}(x, y) \leq M_2 \mathop{b}(x, y) \\
\end{align*}
täytyy osoittaa että näistä seuraa:
$a \buildrel{bL}\over\sim c  \iff \mathop{a}(x, y)/M_3 \leq \mathop{c}(x, y) \leq M_3 \mathop{a}(x, y)$
Lähdetään liikkeelle:
\begin{align*}
b \buildrel{bL}\over\sim c \iff \mathop{b}(x, y)/M_2 \leq \mathop{c}(x, y) \leq M_2 \mathop{b}(x, y) \\
\text{Koska } a(x, y) /M_1 \leq b(x, y) \text{ ja } b(x, y) \leq M_1 a(x, y) \\
\frac{\mathop{a}(x, y)}{M_1}/M_2 \leq \mathop{c}(x, y) \leq M_2 M_1 \mathop{a}(x, y) \\
M_3 = M_1 \cdot M_2
\end{align*}
Epäyhtälö pätee ja vakio on löydettävissä joten väite on tosi.
\end{enumerate}
\bigskip

\noindent  2:2. Olkoon 
\[
C([0,1]) = \{f: [0,1] \to \mathbb R \ | \  f \textrm{ on jatkuva välillä } [0,1]\}.
\]
Pidetään tunnettuna että 
\[
f \mapsto \vert f\vert_{\infty} = \max_{t \in [0,1]} \vert f(t) \vert, \   \textrm{ ja }  f \mapsto \vert f\vert_1 = \int_0^1 \vert f(t) \vert dt
\]
ovat vektoriavaruuden $C([0,1])$ normeja. Näytä: 
\begin{enumerate}
\item $\vert f\vert_{\infty} \le \vert f\vert_1$, kun $f \in C([0,1])$.

\textbf{Ratkaisu: }
Koska $f$ on jatkuva välillä $[0, 1]$ se saavuttaa jonkin suurimman arvon tällä välillä. Tämä tarkoittaa että voidaan määrätä $\max_{t \in [0,1]} \vert f(t) \vert  = M$. Toisaalta tämä arvo sisältyy myös integraaliin $\int_0^1 \vert f(t) \vert dt$, joten voidaan sanoa $M \leq \int_0^1 \vert f(t) \vert dt$. Tästä seuraa että $\vert f\vert_{\infty} \le \vert f\vert_1$. \qed
\smallskip

\item Normit $\vert \cdot \vert_{\infty}$ ja $\vert \cdot \vert_1$  eivät ole bilipschitz-ekvivalentit 
vektoriavaruudessa $C([0,1])$. \textit{Vihje}:  sopivia polynomeja  tai paloittain affiineja kuvauksia.
\end{enumerate}
\textbf{Ratkaisu:}

Jotta normit olisivat bilipschitz ekvivalentit, täytyisi olla olemassa sellaiset a ja b, $0 \le a \leq b$, joilla kaikilla $f \in C([0,1])$ pätisi seuraava:
\begin{align*} 
a |(f)| &\leq |(f)| \leq b |(f)| \\
 \iff a | f - g |_\infty &\leq |f - g|_1 \leq b| f - g |_\infty \\
 \iff a \max_{t \in [0,1]} \vert f(t) - g(t) \vert &\leq \int_0^1 \vert f(t) - g(t) \vert dt \leq b \max_{t \in [0,1]} \vert f(t) - g(t) \vert
\end{align*}
Tarkastellaan funktioita $f(t) = t^n$ ja $g(t) = 0$ $n \in \mathbb{N}$

\begin{align*}
    a \max_{t \in [0,1]} \vert f(t) - g(t) \vert &\leq \int_0^1 \vert f(t) - g(t) \vert dt \leq b \max_{t \in [0,1]} \vert f(t) - g(t) \vert \\
    a \max_{t \in [0,1]} \vert t^n \vert &\leq \int_0^1 \vert t^n \vert dt \leq b \max_{t \in [0,1]} \vert t^n \vert \\
    a  & \leq \int_0^1 t^n dt \leq b \\
    a  & \leq \frac{1}{n+1} \leq b
\end{align*}
Havaitaan että kun $n$ on iso, integraali "liiskaantuu" kohti nollaa, rajoittaen $a$:n arvoja. 

\bigskip

\noindent  2:3. Pidetään tunnettuna, että 
\[
e(s,t) = \frac{\vert s-t\vert}{1+\vert s-t\vert}, \quad s,t \in \mathbb R,
\]
on metriikka reaalisuoralla $\mathbb R$.
\begin{enumerate}
\item Onko metriikka $e$ ekvivalentti 
reaalisuoran tavallisen metriikan $(s,t) \mapsto \vert s-t\vert$ kanssa?
\textit{Vihje}: HT 1:6 auttaa tässä.

\textbf{Ratkaisu:}
Metriikat ovat ekvivalentit jos ne määräävät samat avoimet joukot. 

Suurilla etäisyyksillä metriikka e toimii käytännössä täysin samoin kuin tavallinen metriikka. Pienillä etäisyyksillä metriikka poikkeaa enemmän, mutta se on kuitenkin jatkuva, joten se määrää samat avoimet joukot.

Tarkastellaan kuulia kummallakin metriikalla:
$a = (a), r_1 = 1$
$B_t = B_{d_t}(a, r_1) = \{x \in R^2 : d_t(x, a) < r_1\}$

$B_e = B_{d_e}(a, r_2) = \{x \in R^2 : e(x, a) < r_2\} $
Oletetaan että sellainen löytyy, ja tarkastellaan yhtälöä:
\begin{align*}
    d_1(x, a) < 1 &\iff e(x, a) < r_2 \\
    \vert x - a \vert < 1 &\iff \frac{\vert x-a\vert}{1+\vert x-a\vert} < r_2\\
\end{align*}    
$r_2$ voidaan määrittää joten metriikat ovat ekvivalentit.
\item Ovatko metriikat $e$ ja $\vert \cdot \vert$ bilipschitz-ekvivalentit reaalisuoralla?

\textbf{Ratkaisu:}
\end{enumerate}
Tarkastellaan:
$|t - s|/M \leq \frac{\vert t-s\vert}{1+\vert t-s\vert} \leq M |t - s|$
Yhtälö on totta tapauksessa $t-s = 0$. Muiden tapauksien laskemista varten jaetaan yhtälö $|t-s|$:llä.
$1/M \leq \frac{1}{1+\vert t-s\vert} \leq M $
Tästä voidaan rajata M arvo, joten metriikat ovat bilipschitz ekvivalentteja.
\bigskip

\noindent  2:4.  Olkoon $e$ lukusuoran $\mathbb R$ diskreetti 
$\{0,1\}$-metriikka, sekä 
\[
d(x,y) = \vert x_1 - y_1\vert + e(x_2, y_2), \textrm{ kun }  x = (x_1,x_2), y = (y_1,y_2) 
\in \mathbb R^2.
\]
Kohdan [V,10.9] nojalla $d$ on tason $\mathbb R^2$ metriikka.  Olkoot 
\[
z_n = (1/n,1/n) \textrm{ ja } v_n = (1/n,0), \quad n \in \mathbb N.
\]
 Tutki, suppenevatko pistejonot 
$(z_n)$ ja $(v_n)$ metrisessä avaruudessa $(\mathbb R^2,d)$. Perustele huolellisesti.

\textbf{Ratkaisu: }
Tarkastellaan pistejonoa $(z_n)$.
Lauseen 11.3 mukaan $z_n \rightarrow a$, eli jonon suppeneminen kohti pistettä $a$, on yhtäpitävää ehdon 2 kanssa.

Jos U on a:n ympäristö niin $z_n \notin U$ vain äärellisen monella indeksillä $n \in N$. 

Tutkitaan pisteen $a$ ympäristöä. On $B(a, 1/2)$, johon kuuluu vain yksi piste jonosta. Tällöin siihen ei kuulu äärettömän montaa muuta pistettä, joten jono ei suppene. % TODO: periustele sillä että ainoastaan pisteet joiden x_2 = y_2 ovat "vierekkäin"

Tarkastellaan pistejonoa $v_n$. Veikataan että se suppenee pistettä $a=(0, 0)$ kohti. 

Mukaillaan 11.3 (3):sta. Eli $d(v_n, a) < \varepsilon$ kun $n \geq n_0$.

Tarkastellaan etäisyyttä $d(v_n, a)$. Valitaan että $n_0 > 1 / \varepsilon$ 

$d(v_n, a) = |1/n - 0| + e(0, 0) = 1/n \leq 1/n_0 < \varepsilon$.

Tällöin epäyhtälö toteutuu joten jono suppenee pisteeseen a.

\bigskip

 2:5. Olkoon $\vert \cdot \vert$ reaalisuoran itseisarvometriikka, sekä $e$ reaalisuoran diskreetti $\{0,1\}$-metriikka. Varustetaan taso 
$\mathbb R^2 = \mathbb R \times \mathbb R$ metriikoilla 
\[
d_1(x,y) = \vert x_1 - y_1\vert + e(x_2,y_2),  \quad d_2(x,y) =   e(x_1,y_1) + \vert x_2 - y_2\vert ,
\]
 kun $x = (x_1,x_2), y = (y_1,y_2) \in \mathbb R^2$. Näytä, että 

\begin{enumerate}
\item kuvaus $f$ on homeomorfismi $(\mathbb R^2,d_1) \to (\mathbb R^2,d_2)$, kun asetetaan 
\[
f(x_1,x_2) = (x_2,x_1), \quad (x_1,x_2) \in \mathbb R^2.
\]
\textbf{Ratkaisu: }
$f$ on selkeästi jatkuva, sillä projektiokuvaukset $f_1(x, y) = y$  ja $f_2(x, y) = x$ ovat jatkuvia. Lisäksi kuvauksen $f$ käänteiskuvaus on kuvaus $f$. $f \circ f (x, y) = f(y, x) = (x, y)$. Täten $f$ on homeomorfismi.

\smallskip

\item Metriikat $d_1$ ja $d_2$ eivät ole ekvivalentit tasossa $\mathbb R^2$.  \textit{Vihje}: vertaile vastaavia avoimia kuulia.

\end{enumerate}
\textbf{Ratkaisu: } Tarkastellaan kuulia kummassakin avaruudessa.
$a = (1, 0), r_1 = 1$
$B_1 = B_{d_1}(a, r_1) = \{x \in R^2 : d_1(x, a) < r_1\}$

$B_2 = B_{d_2}(a, r_2) = \{x \in R^2 : d_2(x, a) < r_2\} $
Jotta avaruudet olisivat ekvivalentit,  tulisi voida määritellä arvo säteelle $r_2$ joilla kumpikin kuulaympäristö on sama.
Oletetaan että sellainen löytyy, ja tarkastellaan yhtälöä:
\begin{align*}
    d_1(x, a) < 1 &\iff d_2(x, a) < r_2 \\
    \vert x_1 - 1 \vert + e(x_2,0) < 1 &\iff e(x_1,1) + \vert x_2 - 0 \vert < r_2 \\
\end{align*}
Nähdään että vasen puoli pätee ainoastaan jos $e(x_2, 0) < 1$, joka on mahdollista ainoastaan pisteissä joissa $x_2 = 0$. (Toisin sanottuna $B_1$ sisältää ainoastaan pisteistä janalla $(x_1, 0)$) Tarkastellaan näitä pisteitä, eli asetetaan $x_2=0$.
\begin{align*}
    \vert x_1 - 1 \vert + e(0,0) < 1 &\iff e(x_1,1) + \vert 0 - 0 \vert < r_2 \\
    \vert x_1 - 1 \vert  < 1 &\iff e(x_1,1) < r_2 \\
\end{align*}
Vasen puoli on totta mikäli $0 < x_1 < 2$. Oikealla puolella mikäli $r_2 > 1$ kaikki pisteet sisältyvät palloon, ja mikäli $r_2 < 1$ pienemmyys pätee ainoastaan mikäli $x_1 = 1$. Eli piste $(1/2, 0)$ sisältyy $B_1$, ja sitä ei ole mahdollista sisällyttää $B_2$  sisällyttämättä samalla monia jotka eivät sisälly $B_2$. Täten avaruudet eivät ole ekvivalentit. \qed
\bigskip

\noindent  2:6. Olkoon $(X,d)$ ja $(Y,d')$ metrisiä avaruuksia, sekä varustetaan tuloavaruus $Z = X \times Y$ summametriikalla 
\[
e_1(z,z') = d(x,x') + d'(y,y'),  \textrm{ kun }  z = (x,y),\  z' = (x',y') \in Z.
\]
Olkoon lisäksi $p_1: Z \to X$ projektiokuvaus $p_1(z) = x$,  kun $z = (x,y) \in Z$. Näytä: jos $U \subset Z$ on avoin joukko, niin 
kuva $p_1(U)$ on avoin joukko avaruudessa $X$. \textit{Vihje}:  piirrä kuva tason tapauksessa $X = Y = \mathbb R$.

\smallskip

\textit{Terminologia}: projektio $p_1$ on \textit{avoin} kuvaus  $X \times Y \to X$. 

\bigskip

Kaikissa tehtävissä vastaukset tulee \textbf{perustella}. Pelkkä vastaus ei riitä pisteisiin, ellei tehtävässä erityisesti niin mainita.
Tehtävien arvostelussa käytetään seuraavaa asteikkoa: 
\begin{itemize}
\item [0 p.] Ei ratkaisua tai ratkaisussa ei oikeita elementtejä.
\item [1 p.] Ratkaisussa oikeita elementtejä, mutta kokonaisuutena puutteellinen tai vain osa tehtävästä on ratkaistu.
\item [2 p.] Ratkaisu (lähes) oikein.
\end{itemize}

\end{document} 

